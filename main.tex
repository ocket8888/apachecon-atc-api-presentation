\documentclass[10pt]{beamer}
\usetheme[
%%% option passed to the outer theme
%    progressstyle=fixedCircCnt,   % fixedCircCnt, movingCircCnt (moving is deault)
]{Feather}

% If you want to change the colors of the various elements in the theme, edit and uncomment the following lines

% Change the bar colors:
%\setbeamercolor{Feather}{fg=red!20,bg=red}

% Change the color of the structural elements:
%\setbeamercolor{structure}{fg=red}

% Change the frame title text color:
%\setbeamercolor{frametitle}{fg=blue}

% Change the normal text color background:
%\setbeamercolor{normal text}{fg=black,bg=gray!10}

\usepackage[utf8]{inputenc}
\usepackage[english]{babel}
\usepackage[T1]{fontenc}
\usepackage{helvet}
\usepackage{xcolor}
\usepackage[newfloat]{minted}

\definecolor{inlinecodecolor}{HTML}{800000}
\newcommand{\code}[1]{\texorpdfstring{\texttt{\color{inlinecodecolor}#1}}{#1}}

% colored hyperlinks
\newcommand{\chref}[2]{
	\href{#1}{{\usebeamercolor[bg]{Feather}#2}}
}

\title[Traffic Ops API Design] % [] is optional - is placed on the bottom of the sidebar on every slide
{ % is placed on the title page
			\textbf{Traffic Ops API Design}
}

\subtitle[]
{
			\textbf{v. 1.0.0}
}

\author[ocket8888]
{
	ocket8888 \\
	{\ttfamily ocket8888@apache.org}
}

\date{\today}

%-------------------------------------------------------
% THE BODY OF THE PRESENTATION
%-------------------------------------------------------

\begin{document}

\defverbatim[colored]\exampleDSFullPost{
\begin{minted}[tabsize=2]{json}
{
	"xmlId": "demo1",
	"displayName": "demo1",
	"longDescription": "some extremely long string",
	"rawRemapText": "something that could break the CDN",
	"consistentHashQueryParams": ["xxx", "yyy", "zzz"]
}
\end{minted}
}

\defverbatim[colored]\exampleDSPatch{
\begin{minted}[tabsize=2]{json}
{
	"consistentHashQueryParams": ["test", "quest"]
}
\end{minted}
}

\defverbatim[colored]\urlkeysGenResponse{
\begin{minted}[tabsize=2]{json}
{
	"response": "Successfully generated and stored keys"
}
\end{minted}
}

\defverbatim[colored]\urlkeysNewResponse{
\begin{minted}[tabsize=2]{json}
{ "alerts": [{
	"level": "success",
	"text": "Successfully generated and stored keys"
}],
"response": {
	"key6":"JhGdpw5X9o8TqHfgezCm0bqb9SQPASWL",
	"key0":"D4AYzJ1AE2nYisA9MxMtY03TPDCHji9C",
	"key3":"W90YHlGc_kYlYw5_I0LrkpV9JOzSIneI",
	"key2":"0qgEoDO7sUsugIQemZbwmMt0tNCwB1sf",
	"key4":"aFJ2Gb7atmxVB8uv7T9S6OaDml3ycpGf",
	"key1":"wnWNR1mCz1O4C7EFPtcqHd0xUMQyNFhA",
	"key5":"SIwv3GOhWN7EE9wSwPFj18qE4M07sFxN"
}}
\end{minted}
}

%-------------------------------------------------------
% THE TITLEPAGE
%-------------------------------------------------------

{\1% % this is the name of the PDF file for the background
\begin{frame}[plain,noframenumbering] % the plain option removes the header from the title page, noframenumbering removes the numbering of this frame only
	\titlepage % call the title page information from above
\end{frame}}


\begin{frame}{Content}{}
\tableofcontents
\end{frame}

%-------------------------------------------------------
\section{Introduction}
%-------------------------------------------------------
\subsection{Anatomy of a meeting}
\begin{frame}{Introduction}{Anatomy of a meeting}
%-------------------------------------------------------

	\begin{itemize}
		\item Roughly 20 meetings
		\item Consistent attendance from about 4 people
		\item 1 hour duration
		\item Living document as a \chref{https://github.com/ocket8888/trafficcontrol/pull/14}{pull request}
		\item Two merged \chref{https://github.com/apache/trafficcontrol/tree/master/blueprints}{"blueprints"}
	\end{itemize}
\end{frame}

%-------------------------------------------------------
\section{Motivations}
\begin{frame}{Motivations}{}
%-------------------------------------------------------
	There are two main concerns addressed in the API document/meetings:
	\pause
	\begin{block}{API Best Practices}
	\begin{itemize}
		\item Proper use of HTTP request methods
		\item Clarification of status codes and their purposes
		\item No more relationships-as-objects or "join table endpoints"
		\item Difference between "alerts" and a response
	\end{itemize}
	\end{block}

	\begin{block}{Data Model}
	\begin{itemize}
		\item Separation between database and API
		\item Prefer name-based identification
		\item Reduce harmful reflection
		\item Reduce field "abuse"
	\end{itemize}
	\end{block}
\end{frame}


%-------------------------------------------------------
\subsection{API Best Practices}
\begin{frame}{API Best Practices}{Proper use of HTTP request methods}
%-------------------------------------------------------
	Request methods should be indicative of the action being performed - i.e. no
	more endpoints like \chref{https://traffic-control-cdn.readthedocs.io/en/latest/api/v1/deliveryservices\_xmlid\_xmlid\_sslkeys\_delete.html}{\code{GET /deliveryservices/xmlId/\{\{xmlID\}\}/sslkeys/delete}}.\\
	\code{GET} must \emph{never} modify API objects in any way.\\
	Recommend \code{PATCH} support - objects can be huge.
	\begin{overprint}
		\onslide<1>
		\begin{block}{Example \code{PUT} request for updating a Delivery Service}
			\exampleDSFullPost{}
		\end{block}
		\onslide<2>
		\begin{block}{Example \code{PATCH} request to update one field of a Delivery Service}
			\exampleDSPatch{}
		\end{block}
	\end{overprint}
\end{frame}

%-------------------------------------------------------
\begin{frame}{API Best Practices}{Clarification of status codes and their purposes}
%-------------------------------------------------------
	\begin{itemize}
		\item<1-> Use \code{403 Forbidden} for out-of-Tenant access, not \code{404 Not Found}
		\item<2-> Always use \code{500 Internal Server Error} to hide error details from clients
		\item<3-> \code{201 Created} for object creation
	\end{itemize}
\end{frame}

\begin{frame}{API Best Practices}{No more relationships-as-objects or "join table endpoints"}
	If an object contains a set, map, or list of other objects, they should be
	properties of that parent object - \emph{not} manipulated by a separate
	endpoint.
	\begin{itemize}
		\item<2-> \chref{https://traffic-control-cdn.readthedocs.io/en/latest/api/v1/servers\_id\_deliveryservices.html}{\code{/servers/\{\{ID\}\}/deliveryservices}}
		\item<3-> \chref{https://traffic-control-cdn.readthedocs.io/en/latest/api/v1/deliveryservices\_id\_servers.html}{\code{/deliveryservices/\{\{ID\}\}/servers}}
		\item<4-> \chref{https://traffic-control-cdn.readthedocs.io/en/latest/api/v1/deliveryservices\_xmlid\_servers.html}{\code{/deliveryservices/\{\{xmlID\}\}/servers}}
		\item<5-> \chref{https://traffic-control-cdn.readthedocs.io/en/latest/api/v1/deliveryservice\_server\_dsid\_serverid.html}{\code{/deliveryservice\_server/\{\{DSID\}\}/\{\{serverID\}\}}}
		\item<6-> \chref{https://traffic-control-cdn.readthedocs.io/en/latest/api/v1/deliveryserviceserver.html}{\code{/deliveryserviceserver}}
		\item<7-> \chref{https://traffic-control-cdn.readthedocs.io/en/latest/api/v1/deliveryservices_id_servers_eligible.html}{\code{/deliveryservices/\{\{ID\}\}/servers/eligible}}
	\end{itemize}
\end{frame}

\begin{frame}{API Best Practices}{Difference between "alerts" and a response}
	\begin{overprint}
		\onslide<1>
			\begin{block}{API v1 response to URL key generation endpoint}
				\urlkeysGenResponse{}
			\end{block}
		\onslide<2>
			\begin{block}{Response conforming to API guidelines}
				\urlkeysNewResponse{}
			\end{block}
	\end{overprint}
\end{frame}

\subsection{Data Model}
\begin{frame}{Data Model}{Separation between database and API}
Exposing database tables directly inevitably leads to guidelines violations and
an overall more fragile design.\\
Endpoints should be designed to serve a purpose, not expose some data\footnote{
	Although that could be the purpose.
}.
\end{frame}

\begin{frame}{Data Model}{Prefer name-based identification}
\begin{itemize}
	\item Names often need to be unique anyway
	\item Much easier to remember
	\item "N+1" query problem
\end{itemize}
\end{frame}

\begin{frame}{Data Model}{Reduce harmful reflection}
Customizable types expose database structure - versioning nightmare
\begin{block}{How to find out if a server is an Edge-tier cache server}
	\begin{enumerate}
		\item<1-> Check its Type
		\begin{enumerate}
			\item<2-> Look up Type by ID
			\item<3-> Check if its Type's Name is the type "EDGE"\footnote{This exists by default in new installations/upgrades - but it can be deleted!}
			\item<4-> Check if its Type's Name matches the pattern \code{\^{}EDGE\_.*}
		\end{enumerate}
		\item<5-> Check its Profile
		\begin{enumerate}
			\item<6-> Look up its Profile by ID
			\item<7-> Check its Profile's Type (same procedure as before - but looking for \code{ATS\_PROFILE})
			\item<8-> Check if its Profile's Name matches the pattern \code{\^{}EDGE\_.*}
		\end{enumerate}
	\end{enumerate}
\end{block}
\end{frame}

\begin{frame}{Data Model}{Reduce field "abuse"}
"description" fields regularly house business logic parsed by third-party tools
\end{frame}

\section{Considerations}
\begin{frame}{Considerations}{...and learned lessons}
\begin{itemize}
	\item Breaking changes - e.g. \code{201 Created} considered a failure
	\item Satisfy use cases - e.g. tags to help with reflection and field abuse
	\item Introduce changes in small chunks
	\item Knowing a thing is bad isn't enough to get rid of it - like \code{ANY\_MAP}
\end{itemize}
\end{frame}

{\1
\begin{frame}[plain,noframenumbering]
	\finalpage{Thank you for listening}
\end{frame}}

\end{document}
